\documentclass[12pt]{article}
\title{Energiegewinnung aus Abwasser in hohen Gebäuden}

\begin{document}

\subsection*{Energiegewinnung aus Abwasser in hohen Gebäuden}

Weltweit wachsen Städte immer mehr in die Höhe. Um in hohen Gebäuden Trinkwasser in die oberen Stockwerke zu pumpen, wird viel Energie benötigt. Das entstehende Abwasser hat eine dementsprechend hohe potentielle Energie, die ungenutzt bleibt, wenn das Wasser einfach zurück in die Kanalisation fliest. Meistens muss es, bevor es zurück in die Kanalisation geleitet werden kann, sogar noch abgebremst werden, wobei die Energie in Form von Wärme verloren geht. 

Um Energie zu sparen, soll das Abwasser durch eine Turbine geführt werden, die einen Strom erzeugenden Generator antreibt. Damit können Wasserpumpen betrieben werden, die Frisches Trinkwasser in die Oberen Stockwerke pumpen. Alternativ dazu kann der Strom auch in das Stromnetz zurückgespiesen werden. 

Im Rahmen des Pro1E wollen wir ein solches Abwasser - Kleinkraftwerk unter den Aspekten der Machbarkeit, Wirtschaftlichkeit und des Umweltschutzes untersuchen.  

\end{document}
