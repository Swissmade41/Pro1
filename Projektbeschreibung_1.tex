\documentclass[12pt]{article}
\title{Energiegewinnung aus Abwasser in hohen Gebäuden}

\begin{document}

\subsection*{Energiegewinnung aus Abwasser in hohen Gebäuden}

Weltweit wachsen Städte immer mehr in die Höhe. Um in hohen Gebäuden Trinkwasser in die oberen Stockwerke zu pumpen, wird viel Energie benötigt. Das entstehende Abwasser hat eine dementsprechend hohe potentielle Energie, die ungenutzt bleibt, wenn das Wasser zurück in die Kanalisation fliest. Zudem muss das Wasser meistens noch abgebremst werden, bevor es zurück in die Kanalisation geleitet werden kann. Dabei geht die Energie in Form von Wärme verloren. 

Um Energie zurück zu gewinnen, soll das Abwasser durch eine Turbine geführt werden, die einen Generator antreibt. Damit kann der Strom zurück zu den Wasserpumpen geführt werden, die frisches Trinkwasser in die oberen Stockwerke pumpen. Alternativ kann der Strom auch in das Stromnetz zurückgespeist werden. 

Im Rahmen des Pro1E wollen wir ein solches Abwasser - Kleinkraftwerk unter den Aspekten der Machbarkeit, Wirtschaftlichkeit und des Umweltschutzes untersuchen.  

\end{document}
