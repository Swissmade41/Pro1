\section{Einleitung}
Weltweit wachsen Städte immer mehr in die Höhe. Um in hohen Gebäuden Trinkwasser in die oberen Stockwerke zu pumpen, wird viel Energie benötigt. Das entstehende Abwasser hat eine dementsprechend hohe potentielle Energie, die ungenutzt bleibt, wenn das Wasser zurück in die Kanalisation fliesst. Zudem muss das Wasser bei grosser Fallhöhe noch abgebremst werden, bevor es zurück in die Kanalisation geleitet werden kann. Dabei geht die Energie in Form von Wärme verloren.
Um Energie zurück zu gewinnen, soll das Abwasser durch eine Turbine geführt werden, die einen Generator antreibt. Damit kann der Strom zurück zu den Wasserpumpen geführt werden, die frisches Trinkwasser in die oberen Stockwerke pumpen. Alternativ kann der Strom auch in das Stromnetz zurückgespeist werden.\\
Im Rahmen des Pro1E wollen wir ein solches Abwasser - Kleinkraftwerk unter den Aspekten der Machbarkeit, Wirtschaftlichkeit und des Umweltschutzes untersuchen.\\
Die Studierenden werden im Projekt 1 (pro1E) für den Studiengang Elektro- und Informatitonstechnik von drei Dozenten der Fachhochschule Nordwestschweiz (FHNW) unterstützt. Pascal Buchschacher informiert über Projektmanagement allgemein, Anita Gertiser vermittelt den Studenten die richtige Kommunikation innerhalb des Teams und Felix Jenni steht als Ansprechpartner für Fragen technischer Natur zur Verfügung.





