\section{Übersicht} \label{sec:uebersicht}
\subsection{Ausgangslage}
Der Auftrag des Projekts 1 ist der Ersatz von Fossilen Ressourcen durch Elektrizität an einem ausgewählten Produkt. Das Team 4 hat sich das Ziel gesetzt, Lösungen zu finden, um die potentielle Energie des fallenden Abwassers in Hochhäusern und Wolkenkratzern in elektrische Energie umzuwandeln. Wird diese Energie zurück ins Gebäude gespeist, leistet unsere Lösung zwar keinen Ersatz von fossilen Ressourcen, aber einen Beitrag zur Reduktion des fossilen oder elektrischen Energieverbrauchs innerhalb von Gebäuden.
Durch die Recherchearbeit konnte das Team drei potentielle Lösungen finden, die nun in diesem technischen Teil des Pflichtenhefts weiter ausgearbeitet werden.

\subsection{Ziele}
Folgende Ziele hat sich das Team 4 in Absprache mit Herrn Jenni gesetzt:
\begin{table}[H]
\begin{tabular}{lll}
Vorhandenes Potential mit Hilfe eines Modellhochhauses berechnen
Wirtschaftlichkeit untersuchen
\end{tabular}
\end{table}

\subsection{Nicht-Ziele}
Folgende Ziele wurden nicht verfolgt:
\begin{table}[H]
\begin{tabular}{lll}
Material und Form der Bestandteile bestimmen
Absprache mit Architekten
Realisierung
Respektierung der Normen
\end{tabular}
\end{table}
\subsection{Anforderungen}
Anforderungen an die potentielle Lösung sind die folgenden:\\
\begin{table}[H]
\begin{tabular}{lll}
Anforderung											&Mindest-Anforderung																		&Wunsch-Anforderung\\
\textbf{1. Technische Anforderungen}					&																						&\\
\qquad 1.1. Komplexität								&																						&
\qquad 1.2. Wirkungsgrad								&> 0.85																					&>0.9\\
\qquad 1.3. Stabilität								&																						&\\
\textbf{2. Finanzielle Anforderungen}				&																						&\\
\qquad 2.1. Amortisationszeit						&5 Jahre																					&3 Jahre\\
\textbf{3. Praktische Anforderungen}					&																						&\\
Wartungs	intervall									&>5 Jahre																				&10 Jahre\\          
\end{tabular}
\end{table}
