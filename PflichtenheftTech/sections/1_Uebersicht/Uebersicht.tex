\section{Übersicht} \label{sec:uebersicht}
\subsection{Ausgangslage}
Der Auftrag des Projekts 1 ist der Ersatz von fossilen Ressourcen durch Elektrizität an einem ausgewählten Produkt. Das Team 4 hat sich das Ziel gesetzt, Lösungen zu finden, um die potentielle Energie des fallenden Abwassers in Hochhäusern und Wolkenkratzern in elektrische Energie umzuwandeln. Wird diese Energie zurück ins Gebäude gespeist, leistet unsere Lösung zwar keinen Ersatz von fossilen Ressourcen, aber einen Beitrag zur Reduktion des fossilen oder elektrischen Energieverbrauchs innerhalb von Gebäuden.
Durch die Recherchearbeit konnte das Team 4 potentielle Lösungen finden, die nun in diesem technischen Teil des Pflichtenhefts weiter ausgearbeitet und anhand eines Modell-Wolkenkratzers (siehe Subsection \ref{subsec:modell}) miteinnder verglichen werden.
\renewcommand\arraystretch{1.5}
\definecolor{hr}{RGB}{255 200 200}
\definecolor{hgr}{RGB}{200 255 200}
\newcommand{\T}{\rule{0pt}{3ex}}       % Top strut
\newcommand{\B}{\rule[-2.5ex]{0pt}{0pt}} % Bottom strut
\subsection{Ziele}
Folgende Ziele hat das Team 4 festgelegt:
\newcommand{\HY}{\hyphenpenalty = 25\exhyphenpenalty = 25}
\begin{table}[H]
\small
\begin{tabular}{>{\HY\RaggedRight}p{5cm} >{\HY\RaggedRight}p{6.5cm} >{\HY\RaggedRight}p{3cm}}
\hline
\textbf{Zielkriterium}					&\textbf{Zielvariable}									&\textbf{Randbedingung}\\
\hline
\rowcolor{hellgrau}
\multicolumn{3}{l}{\textbf{1. Elektrotechnik}}\T\\
1.1. Wirkungsgrad							&Gesamtwirkungsgrad [\%]								&>70\%\\
1.2. Leistung								&Gesamtleistung [kWh]								&möglichst hoch\B\\
\rowcolor{hellgrau}
\multicolumn{3}{l}{\textbf{2. Abwassertechnik}}\T\\
%2.1. Verstopfungssicherheit					&Verstopfungswahrscheinlichkeit						&möglichst klein\\
2.1. Niedriger Platzverbrauch				&benötigte Grundfläche [m\textsuperscript{2}]			&möglichst klein\B\\
%2.3. Wartung									&Wartungsinterval								&möglichst lange\\		
\rowcolor{hellgrau}
\multicolumn{3}{l}{\textbf{3. Allgemein}}\T\\			
3.1. Schlichtheit							&Anzahl verschiedenartiger Bestandteile				&möglichst niedrig\B\\								
\hline
\end{tabular}
\end{table}
%\rowcolor{grau}
%\multicolumn{4}{l}{\textbf{3. Finanzen evt weglassen}}\\
%3.1. Materialkosten							&Kosten aller Bestandteile [CHF]					&liegen im Bereich einer Solaranlage mit vergleichbarer Leistung	&\\
%3.2. Rendite									&Amortisationszeit	[Jahre]						&																&\\
%3.3. Montagekosten							&Kosten des Einbaus aller Bestandteile [CHF]		&liegt im Bereich einer Solaranlage mit vergleichbarer Leistung	&\\
\newpage
\subsection{Nicht-Ziele}
Da das Projekt 1 als Übung für die Abwicklung eines Projekts dient, werden sämtliche praktische Arbeiten wie Realisierung, Validierung und Projektabschluss nicht umgesetzt.\\
Auch der juristische Teil wird im Projekt 1 nicht beachtet.
Folgende Nicht-Ziele wurden definiert:
\begin{table}[H]
\small
\begin{tabular}{ll}
\hline
\textbf{Nicht-Zielkriterium}				&\textbf{Nicht-Zielvariable}											\\
\hline
\rowcolor{hellgrau}
\textbf{1. Planung}						&																	\T\\
										&Respektierung der Normen											\\
										&Berechnung der Lärmbelastung										\B\\	
\rowcolor{hellgrau}
\textbf{2. Realisierung}					&																	\T\\
										&Erstellen und Testen eines Prototyps								\B\\
\hline
\end{tabular}
\end{table}
