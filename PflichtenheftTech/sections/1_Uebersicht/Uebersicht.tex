\section{Übersicht} \label{sec:uebersicht}
\subsection{Ausgangslage}
Der Auftrag des Projekts 1 ist der Ersatz von Fossilen Ressourcen durch Elektrizität an einem ausgewählten Produkt. Das Team 4 hat sich das Ziel gesetzt, Lösungen zu finden, um die potentielle Energie des fallenden Abwassers in Hochhäusern und Wolkenkratzern in elektrische Energie umzuwandeln. Wird diese Energie zurück ins Gebäude gespeist, leistet unsere Lösung zwar keinen Ersatz von fossilen Ressourcen, aber einen Beitrag zur Reduktion des fossilen oder elektrischen Energieverbrauchs innerhalb von Gebäuden.
Durch die Recherchearbeit konnte das Team 4 potentielle Lösungen finden, die nun in diesem technischen Teil des Pflichtenhefts weiter ausgearbeitet werden.
\renewcommand\arraystretch{2}
\definecolor{hr}{RGB}{255 200 200}
\definecolor{hgr}{RGB}{200 255 200}
\subsection{Ziele}
Folgende Ziele hat sich das Team 4 gesetzt:
\newcommand{\HY}{\hyphenpenalty = 25\exhyphenpenalty = 25}
\begin{table}[H]
\small
\begin{tabular}{>{\HY\RaggedRight}p{5cm} >{\HY\RaggedRight}p{6.5cm} >{\HY\RaggedRight}p{3cm}}
\hline
\textbf{Zielkriterium}					&\textbf{Zielvariable}									&\textbf{Randbedingung}\\
\hline
\rowcolor{grau}
\multicolumn{3}{l}{\textbf{1. Elektrotechnik}}\\
1.1. Wirkungsgrad							&Gesamtwirkungsgrad [\%]								&>70\%\\
1.2. Leistung								&Gesamtleistung [kWh]								&möglichst hoch\\
1.3. Komplexität								&Anzahl der Bestandteile								&möglichst niedrig\\
\rowcolor{grau}
\multicolumn{3}{l}{\textbf{2. Abwassertechnik}}\\
2.1. Verstopfungsgefahr						&Verstopfungswahrscheinlichkeit						&möglichst klein\\
2.2. Platzbeanspruchung						&Dimension der zusätzlichen Infrastruktur				&möglichst klein\\
2.3. Wartung									&Wartungsinterval									&möglichst lange\\													
\hline
\end{tabular}
\end{table}
%\rowcolor{grau}
%\multicolumn{4}{l}{\textbf{3. Finanzen evt weglassen}}\\
%3.1. Materialkosten							&Kosten aller Bestandteile [CHF]					&liegen im Bereich einer Solaranlage mit vergleichbarer Leistung	&\\
%3.2. Rendite									&Amortisationszeit	[Jahre]						&																&\\
%3.3. Montagekosten							&Kosten des Einbaus aller Bestandteile [CHF]		&liegt im Bereich einer Solaranlage mit vergleichbarer Leistung	&\\
\newpage
\subsection{Nicht-Ziele}
Da das Projekt 1 als Übung für die Abwicklung eines Projekts dient, werden sämtliche praktische Arbeiten wie Realisierung, Validierung und Projektabschluss nicht umgesetzt.\\
Auch der juristische Teil wird im Projekt 1 nicht beachtet.
Folgende Nicht-Ziele wurden definiert:
\begin{table}[H]
\begin{tabular}{ll}
\textbf{Nicht-Zielkriterium}				&\textbf{Nicht-Zielvariable}											\\
\hline
\rowcolor{grau}
\textbf{1. Planung}						&																	\\
										&Respektierung der Normen											\\
										&Machbarkeitsstudie													\\
\rowcolor{grau}
\textbf{2. Realisierung \& Kosten}		&																	\\
										&Einbau und Anschluss der Bestandteile\\
										&Testlauf															\\
										&Lärmbelastung														\\		
\hline
\end{tabular}
\end{table}
