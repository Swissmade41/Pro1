\subsection{Nutzwertanalyse} \label{subsec:nutzwertanlyse}
\renewcommand{\arraystretch}{1.5}
\newcommand{\vtcl}[1]{\rotatebox{90}{#1}}%\newcommand{\vtcl}[1]{\rotatebox{90}{\textbf{#1}}}
%\newcommand{\diagL}[1]{\diagbox{\hspace{#1}}{\hspace{#1}}}
Das Team konnte durch folgende Nutzwertanalyse bestimmen, welches Grobkonzept am ehesten in Fage kommt.
%\begin{table}[H]
%\small
%\begin{tabular}{l|llll|rr}
%&\vtcl{1.1. Wirkungsgrad}&\vtcl{1.2. Leistung}&\vtcl{1.3. Komplexität}&\vtcl{2.1. Platzbedarf}&\vtcl{Total}&\vtcl{Prozent}\\%\vtcl{2.1. Verstopfungsgefahr}&\vtcl{2.2. Platzbedarf}&\vtcl{2.3. Wartung}&\vtcl{Total}&\vtcl{Prozent}\\
%\hline
%1.1. Wirkungsgrad		&\cellcolor{black}	&0.5					&0.5					&0.5					&2.		&13\%\\
%1.2. Leistung			&0.5					&\cellcolor{black}	&1					&1					&4.5		&30\%\\
%1.3. Komplexität			&0.5					&0					&\cellcolor{black}	&0					&1.0		&6.5\%\\
%2.1. Verstopfungsgefahr	&1					&0					&1					&\cellcolor{black}	&0					&1					&3		&20\%\\
%2.1. Platzbedarf			&0.5					&0					&1					&\cellcolor{black}	&3.5		&24\%\\
%2.3. Wartung			&0.5					&0					&0.5					&0					&0					&\cellcolor{black}	&1.0		&6.5\%\\
%\hline
%\multicolumn{7}{c}{}&\textbf{15}&\textbf{100}\%\\
%\end{tabular}
%\end{table}
%\begin{scriptsize}
%Zeile-Kriterium ist wichtiger als Spalten-Kriterium 1\\
%Zeile-Kriterium ist gleich wichtig wie Spalten-Kriteriium 0.5\\
%Zeile-Kriterum ist weniger wichtig wie Spalternkriterium 0\\
%\end{scriptsize}
\newcolumntype{C}[1]{>{\centering}p{#1}}


\renewcommand\arraystretch{1.5}
\newcolumntype{R}[1]{>{\HY\RaggedLeft}p{#1}}
\newcolumntype{L}[1]{>{\HY\RaggedRight}p{#1}}
\renewcommand{\vtcl}[1]{\rotatebox{90}{\textbf{#1}}}
\begin{table}[H]
\small
\rotatebox{90}{
%|R{1.2cm}R{0.4cm}R{0.8cm} \scriptsize
%|R{1cm}R{0.2cm}R{0.5cm} \tiny
\begin{tabular}{lrR{0.8cm}|rrr|rrr|rrr|rrr|}%|R{1cm}R{0.5cm}R{0.8cm}
&&Max&\multicolumn{3}{c}{Grobkonzept 1}&\multicolumn{3}{c}{Grobkonzept 2}&\multicolumn{3}{c}{Grobkonzept 3}&\multicolumn{3}{c}{Grobkonzept 4}\\
\textbf{Zielkriterium}&\vtcl{Gewichtung}&\vtcl{Nutzwert}&\vtcl{Wert}&\vtcl{Erfüllungsgrad}&\vtcl{Nutzwert}&\vtcl{Wert}&\vtcl{Erfüllungsgrad}&\vtcl{Nutzwert}&\vtcl{Wert}&\vtcl{Erfüllungsgrad}&\vtcl{Nutzwert}&\vtcl{Wert}&\vtcl{Erfüllungsgrad}&\vtcl{Nutzwert}\\
\hline
&&&&&&&&&&&&&&\\
\rowcolor{hellgrau}
\multicolumn{3}{l|}{\textbf{Elektrotechnik}}&\multicolumn{3}{r|}{}&\multicolumn{3}{r|}{}&\multicolumn{3}{r|}{}&\multicolumn{3}{r|}{}\\
%										%G1 Wert		%G1 Erf.		G1 Nutz			%G2 Wert		%G2 Erf.		G2 Nutz			%G3 Wert		%G3 Erf.		G2 Nutz			%G4 Wert		%G4 Erf.		G4 Nutz
1.1. Wirkungsgrad	&25\%		&1.25	&32\%		&1			&0.25			&67.2\%		&3			&0.75			&64.1\%		&3			&0.75			&80\%		&4			&1.00\\
1.2. Leistung		&30\%		&1.50	&21.5kWh		&1			&0.33			&44.6kWh		&2			&0.66			&42.6kWh		&2			&0.66			&53.1kWh		&4			&1.20\\
%&&&&&&&&&&&&&&\\
\rowcolor{hellgrau}
\multicolumn{3}{l|}{\textbf{Abwassertechnik}}&\multicolumn{3}{r|}{}&\multicolumn{3}{r|}{}&\multicolumn{3}{r|}{}&\multicolumn{3}{r|}{}\\
%Verstopfungssicherheit&20\%		&1.000	&mässig		&2			&0.40			&mässig		&2			&0.4	0			&mässig		&2			&0.40			&mittel		&3			&0.60\\
2.1. Platzsparung	&25\%		&1.00	&erhöht		&4			&1.00			&mässig		&2			&0.50			&mässig		&2			&0.50			&erhöht		&4			&1.00\\
%Wartung				&6.5\%		&0.325	&5			&2			&0.26			&3			&2			&0.26			&1			&4			&0.20			&1			&4			&0.20\\
\rowcolor{hellgrau}
\multicolumn{3}{l|}{\textbf{Allgemein}}&\multicolumn{3}{r|}{}&\multicolumn{3}{r|}{}&\multicolumn{3}{r|}{}&\multicolumn{3}{r|}{}\\
3.1. Schlichtheit	&20\%		&1.00	&8			&4			&0.8	0			&14			&3			&0.60			&15			&3			&0.60			&10			&4			&0.80\\
&&&&&&&&&&&&&&\\
\hline
Summe				&100.0\%		&5.000	&			&			&2.38			&			&			&2.51			&			&			&2.51			&			&			&4.00\\
Erfüllungsgrad [\%]	&&100.0				&			&			&48				&			&			&50				&			&			&50				&			&			&80\\
\multicolumn{3}{l}{\textbf{Rangfolge}}&\multicolumn{3}{r|}{\textbf{3}}&\multicolumn{3}{r|}{\textbf{2}}&\multicolumn{3}{r|}{\textbf{2}}&\multicolumn{3}{r|}{\textbf{1}}\\
\multicolumn{15}{l}{\textbf{}}\\
\multicolumn{15}{l}{\textbf{}}\\
\end{tabular}
}
\rotatebox{90}{
\scriptsize
\begin{tabular}{lC{1.2cm}C{1.2cm}C{1.2cm}C{1.2cm}C{1.2cm}l}
\multicolumn{7}{c}{\textbf{Erfüllungsgrad}}\\
\hline
&min.&&mittel&&max.&\\
&\textbf{1}&\textbf{2}&\textbf{3}&\textbf{4}&\textbf{5}&Messgrösse\\
\hline
1.1. Wirkungsgrad				&<50&51-60&61-70&71-80&>81&\%\\
1.2. Leistung					&<40&40-44&45-50&51-54&>55&kWh\\
%2.1. Verstopfungssicherheit		&gering&mässig&mittel&erhöht&hoch&a)\\
2.1. Platzsparung				&gering&mässig&mittel&erhöht&gross&Schätzung m\textsuperscript{3}\\
%2.3. Wartung						&52-13&12-6&5-2&1&0&b)\\
3.1. Schlichtheit				&>20&20-16&15-11&10-6&1-5&Anz. versch. Teile\\
\end{tabular}
}
\caption{Nutzwertanalyse}\label{tab:nutzwertanalyse}
\end{table}
