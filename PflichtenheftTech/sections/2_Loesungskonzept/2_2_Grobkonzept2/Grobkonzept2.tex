\subsection{Grobkonzept 2} \label{subsec:grobkonzept2}

\begin{table}[H]
\begin{tabular}{>{\columncolor{hgelb}}l>{\columncolor{dgelb}}l>{\columncolor{hgelb}}llllll>{\columncolor{hgruen}}l>{\columncolor{dgruen}}l>{\columncolor{hgruen}}ll}
%GELBE KOLONNE						BLAUE UND PINKIGE KOLONNE										GRUENE KOLONNE
\titleCell{hgelb}{\textbf{Turbine}}	&&\titleCell{hblau}{\textbf{Elektrotechnik}}					&&\titleCell{hgruen}{\textbf{Abwassertechnik}}&\\
&\textbf{Turbinentyp}				&&&\cC{hblau}	&\cC{dblau}\textbf{Wechselrichter}	&\cC{hblau}	&&&\textbf{Tanks}				&&\\
&Pelton								&&&\cC{hblau}	&\cC{dblau}\textbf{Ventilsteuerung}	&\cC{hblau}	&&&Ablassentile					&&\\
&\textbf{Generatortyp}				&&&\cC{hblau}	&\cC{dblau}							&\cC{hblau}	&&&Entlüftung					&&\\
&Gleichstrom						&&&\titleCell{hblau}{ }											&&&Trichterförmig				&&\\
&\textbf{Anzahl}					&&&&&															&&&Füllstandssensor				&&\\
%ACHTUNG WEISSER ZWISCHENRAUM
&1									&&&\titleCell{hpink}{\textbf{Bedienung}}						&&&Notüberlauf					&&\\
&									&&&\cC{hpink}	&\cC{dpink}\textbf{Anzeige}			&\cC{hpink}	&&&\textbf{Druckleitungen}		&&\\
&									&&&\cC{hpink}	&\cC{dpink}der Füllstände			&\cC{hpink}	&&&Druckfestigkeit >40 bar		&&\\
&									&&&\cC{hpink}	&\cC{dpink}der Aktuellen Leistung	&\cC{hpink}	&&&\textbf{Bypass}				&&\\
&									&&&\cC{hpink}	&\cC{dpink}\textbf{Ventilsteuerung}	&\cC{hpink}	&&&für Tanks					&&\\
&									&&&\cC{hpink}	&\cC{dpink}							&\cC{hpink}	&&&für Turbine					&&\\
\titleCell{hgelb}{ }				&&\titleCell{hpink}{ }											&&\titleCell{hgruen}{ }&
\end{tabular}
\end{table}


Im Grobkonzept 2 soll die Energieausbeutung gesteigert werden, indem das Abwasser zuerst in Tanks gespeichert wird, die all ca. 13 Stockwerke eingebaut sind. Wenn ein Tank voll ist, wird ein Ventil geöffnet und das Wasser fliesst durch eine Druckleitung in den Keller, wo es eine Pelton-Turbine antreibt. Da das Abwasser das Rohr komplett ausfüllt, gibt es keinen Luftwiderstand, der es abbremst. So kann der Wirkungsgrad des Systems verbessert werden. Ausserdem braucht man nur eine Turbine, die von mehreren Tanks gespeist werden kann. Um zu verhindern, dass es in den Tanks zu Ablagerungen kommt, ist der Boden der Tanks trichterförmig. So werden alle Ablageungen beim Öffnen des Ventils weggespült. 
Die baulichen Massnahmen, die nötig sind, um dieses System zu installieren sind beträchtlich. Es müssen Tanks eingebaut werden, Druckleitungen zur Turbine verlegt werden, die im Keller installiert werden muss, und die bestehenden Abwasserleitungen anders verlegt werden, dass sie in die Tanks führen. Somit ist es eher für Neubauten geeignet als zur Nachrüstung.

\textbf{Vorteile:} 									\newline
+	Kein Luftwiderstand (sobald Rohr gefüllt ist)	\newline
+	Nur eine Turbine nötig							\newline
\textbf{Nachteile:}									\newline
-	Braucht viel Platz 								\newline
-	Grössere Bauliche Massnahmen nötig				\newline
-	Verstopfungsgefahr 								\newline
-	Lange Leitungen brauchen länger bis komplett mit Wasser gefüllt, bis dann Luftwiderstand.
