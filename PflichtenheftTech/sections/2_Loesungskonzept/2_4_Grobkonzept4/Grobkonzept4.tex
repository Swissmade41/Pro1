\subsection{Grobkonzept 4} \label{subsec:grobkonzept3}
\begin{table}[H]
\footnotesize
\begin{tabular}{>{\HY\RaggedRight}p{3cm} >{\HY\RaggedRight}p{2.2cm} >{\HY\RaggedRight}p{4cm} >{\HY\RaggedRight}p{3.3cm} >{\HY\RaggedRight}p{1.2cm}}
\hline
	\textbf{Bestandteil}		&\textbf{Typ}			&\textbf{Funktion}									&\textbf{Specs}			&\textbf{Anz.}\\
	\hline
\rowcolor{dgelb}
\multicolumn{5}{l}{\textbf{Stromerzeugung}}\\
	Wasserlift 				& 				&Umwandlung in Rotationsenergie						&							&5	\\
	Generator					&Gleichstrom			&Umwandlung in elektrische Energie					&							&5	\\
\rowcolor{dblau}
\multicolumn{5}{l}{\textbf{Elektrotechnik}}\\
 	Wechselrichter				&						&Einspeisung ins Stromnetz							&							&1	\\
 	Ventilsteuerung				&						&Öffnet/schliesst Ventile je nach Füllstand			&							&1	\\
\rowcolor{dpink}
\multicolumn{5}{l}{\textbf{Bedienung}}\\
 	Anzeige 					&						&zeigt Tankfüllstände und die Generatordaten an 	&							&1	\\
 	Alarmleuchte				&						&Warnt bei zu hochem Füllstand in einem der Tanks 	&							&1	\\
\rowcolor{dgruen}
\multicolumn{5}{l}{\textbf{Abwassertechnik}}\\
	Ablassventil					&						&Entlässt das Abwasser aus dem Tank 				&							&5	\\
\hline
\end{tabular}
\end{table}

Im Grobkonzepts 4 wird die potenzielle Energie des Wassers mit der Wasserlifttechnik ausgenutzt. So wird mittels 

\textbf{Vorteile:}\newline
kostengünstig??			\newline
			\newline
\textbf{Nachteile:}\newline
				\newline
				\newline
				\newline				
\newpage