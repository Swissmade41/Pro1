\subsection{Grobkonzept 1} \label{subsec:grobkonzept1}
\definecolor{dpink}{HTML}{FF13F8}
\definecolor{hpink}{HTML}{FF98FC}
\definecolor{hgruen}{HTML}{9EFFA9}
\definecolor{hblau}{HTML}{51D1FF}
\definecolor{dblau}{HTML}{008BBD}
\definecolor{hgelb}{HTML}{FFFC9E}
\definecolor{dgelb}{HTML}{FFCC67}
\definecolor{dgruen}{HTML}{00AC14}

\newcommand{\titleCell}[2]{\multicolumn{3}{c}{\cellcolor{#1}#2}}
\newcommand{\cC}[1]{\cellcolor{#1}}

\begin{table}[H]
\begin{tabular}{>{\columncolor{hgelb}}l>{\columncolor{dgelb}}l>{\columncolor{hgelb}}llllll>{\columncolor{hgruen}}l>{\columncolor{dgruen}}l>{\columncolor{hgruen}}ll}

%GELBE KOLONNE						BLAUE UND PINKIGE KOLONNE										GRUENE KOLONNE
\titleCell{hgelb}{\textbf{Turbine}}	&&\titleCell{hblau}{\textbf{Elektrotechnik}}						&&\titleCell{hgruen}{\textbf{Abwassertechnik}}&\\
&\textbf{Turbinentyp}					&&&\cC{hblau}	&\cC{dblau}\textbf{Wechselrichter}			&\cC{hblau}	&&&\textbf{Tanks}				&&\\
&Pelton-Turbine							&&&\cC{hblau}	&\cC{dblau}\textbf{Speicherakkus}			&\cC{hblau}	&&&\textbf{Bypass}				&&\\
&\textbf{Generatortyp}					&&&\cC{hblau}	&\cC{dblau}									&\cC{hblau}	&&&	für Turbine									&&\\
&Gleichstrom								&&&\titleCell{hblau}{ }																						&&&								&&\\
&\textbf{Anzahl}							&&&&													&															&&&								&&\\
%ACHTUNG WEISSER ZWISCHENRAUM
&28												&&&\titleCell{hpink}{\textbf{Bedienung}}															&&&								&&\\
&													&&&\cC{hpink}	&\cC{dpink}\textbf{Anzeige}			&\cC{hpink}							&&&\textbf{}						&&\\
&													&&&\cC{hpink}	&\cC{dpink}der Aktuellen Leistung	&\cC{hpink}						&&& 							 	&&\\
&													&&&\cC{hpink}	&\cC{dpink}								&\cC{hpink}								&&&								&&\\
&													&&&\cC{hpink}	&\cC{dpink}\textbf{}							&\cC{hpink}						&&&								&&\\
&													&&&\cC{hpink}	&\cC{dpink}							&\cC{hpink}									&&& 								&&\\
\titleCell{hgelb}{ }				&&\titleCell{hpink}{ }											&&\titleCell{hgruen}{ }&
\end{tabular}
\end{table}


Im Grobkonzept 1 soll die Pelton Turbine direkt in die Fallleitung eingebaut werden. Die Turbine wird so in das vorhandene System integriert ohne das größere Umbauten nötig sind.
In unserem Hochhausmodell (Park Avenue 432) würde immer nach drei Etagen eine Turbine eingebaut werden um die maximale Leistung herausholen zu können. 
Die gewonnene Energie wird mittels eines Wechselrichters transformiert und in das 230V Netz zurückgespiesen. Falls eine Solaranlage mit Wechselrichter vorhanden ist, wäre es die günstigste Option, diesen mitzubenutzen. Das Laufrad der Turbine besteht aus Duplex-Stahl, der gegen Abrasion und korrosive Gase schützt.
Bei Wartungsarbeiten kann das Abwasser mittels Bypass an der Turbine vorbei geleitet werden. So kann die Turbine sicher geöffnet oder ausgewechselt werden.

\textbf{Vorteile:}												\newline
+	kleiner Umbau der vorhandenen Anlage			\newline
+	Kostengünstiger											\newline
	
\textbf{Nachteile:}												\newline
- 	extra Beschichtung vom Laufrad					\newline
-	sporadische Wassermenge							\newline
-	Leistung														\newline


