\subsection{Grobkonzept 1} \label{subsec:grobkonzept1}
\definecolor{dpink}{HTML}{FF13F8}
\definecolor{hpink}{HTML}{FF98FC}
\definecolor{hgruen}{HTML}{9EFFA9}
\definecolor{hblau}{HTML}{51D1FF}
\definecolor{dblau}{HTML}{008BBD}
\definecolor{hgelb}{HTML}{FFFC9E}
\definecolor{dgelb}{HTML}{FFCC67}
\definecolor{dgruen}{HTML}{00AC14}

\newcommand{\titleCell}[2]{\multicolumn{3}{c}{\cellcolor{#1}#2}}
\newcommand{\cC}[1]{\cellcolor{#1}}

%\newcommand{\HY}{\hyphenpenalty = 25\exhyphenpenalty = 25}
\begin{table}[H]
\footnotesize
\begin{tabular}{>{\HY\RaggedRight}p{3cm} >{\HY\RaggedRight}p{2.2cm} >{\HY\RaggedRight}p{4cm} >{\HY\RaggedRight}p{3.3cm} >{\HY\RaggedRight}p{1.2cm}}
\hline
\textbf{Bestandteil}&\textbf{Typ}&\textbf{Funktion}&\textbf{Specs}&\textbf{Anz.}\\
\hline
\rowcolor{dgelb}
\multicolumn{5}{l}{\textbf{Stromerzeugung}}\\
Wasserrad& &Umwandlung in Rotationsenergie&>300W&50\\
Generator&AC&Umwandlung in elektrische Energie&>300W&50\\%Gleichstromwandlung
Generator neu& & &wenn man richtig zaehlt&43\\
\rowcolor{dblau}
\multicolumn{5}{l}{\textbf{Elektrotechnik}}\\
Wechselrichter&&Einspeisung ins Stromnetz&&1\\
\rowcolor{dpink}
\multicolumn{5}{l}{\textbf{Bedienung}}\\
Anzeige&LCD-Display&zeigt Wasserradleistung an&&1\\
Steuerungskasten&&vollautomatische elektrische Steuerung&&1\\
\rowcolor{dgruen}
\multicolumn{5}{l}{\textbf{Abwassertechnik}}\\
Bypass&Absperrklappe&Umleitung für Wartungsarbeiten an dem Wasserrad&&30\\
Bypass neu &&&wenn man richtig zaehlt&43\\
&&&&\\
\hline
\end{tabular}
\end{table}

Im Grobkonzept 1 sollen 50 Wasserräder direkt in die Fallleitung eingebaut werden. Mit jeweils einem Generator pro Wasserrad wird Strom erzeugt. Alle Generatoren sind mit dem Steuerungskasten verbunden. In unserem Hochhausmodell (Park Avenue 432) wird immer nach zwei Etagen ein Wasserrad eingebaut, um die maximale Leistung herausholen zu können. Die gewonnene Energie wird mit einem Wechselrichter transformiert und in das 230V Netz zurückgespiesen. Ist eine Solaranlage mit Wechselrichter vorhanden, könnten bei gemeinsamer Nutzung Kosten gespart werden. 

\subsubsection{Wartung}
Jedes Wasserrad ist mit einem Bypass ausgestattet, mit dem der Abwasserfluss darum herum geführt werden kann. So kann das Wasserrad gewartet oder ersetzt werden.
\bigskip

\textbf{Vorteile:}								\newline
+	kleiner Umbau der vorhandenen Anlage			\newline
+	Kostengünstige Optionen						\newline
+	einfach										\newline
	
\textbf{Nachteile:}								\newline
- 	defekt anfällig								\newline
-	sporadische Wassermenge						\newline
-	kleine Leistung								\newline

