\subsection{Grobkonzept 1} \label{subsec:grobkonzept1}
\definecolor{dpink}{HTML}{FF13F8}
\definecolor{hpink}{HTML}{FF98FC}
\definecolor{hgruen}{HTML}{9EFFA9}
\definecolor{hblau}{HTML}{51D1FF}
\definecolor{dblau}{HTML}{008BBD}
\definecolor{hgelb}{HTML}{FFFC9E}
\definecolor{dgelb}{HTML}{FFCC67}
\definecolor{dgruen}{HTML}{00AC14}

\newcommand{\titleCell}[2]{\multicolumn{3}{c}{\cellcolor{#1}#2}}
\newcommand{\cC}[1]{\cellcolor{#1}}

%\newcommand{\HY}{\hyphenpenalty = 25\exhyphenpenalty = 25}
\begin{table}[H]
\footnotesize
\begin{tabular}{>{\HY\RaggedRight}p{3cm} >{\HY\RaggedRight}p{2.2cm} >{\HY\RaggedRight}p{4cm} >{\HY\RaggedRight}p{3.3cm} >{\HY\RaggedRight}p{1.2cm}}
\hline
\textbf{Bestandteil}&\textbf{Typ}&\textbf{Funktion}&\textbf{Specs}&\textbf{Anz.}\\
\hline
\rowcolor{dgelb}
\multicolumn{5}{l}{\textbf{Stromerzeugung}}\\
Wasserrad& &Umwandlung in Rotationsenergie&>300W&43\\
Generator&AC&Umwandlung in elektrische Energie&>300W&43\\
Gleichrichter&DC/DC Wandler&Gleichstrom zu Gleichstrom&>300W&43\\
DC Bus&&Strom zusammenführen&&1\\
Wechselrichter&&Umwandlungvon DC in AC (230V AC) &&1\\
\rowcolor{dpink}
\multicolumn{5}{l}{\textbf{Kontrollsystem}}\\
PC&&Anlagesteuerung&&1\\
SPS&Beckhof&Analoge und Digitale Aus- und Eingänge&&1\\
\rowcolor{dgruen}
\multicolumn{5}{l}{\textbf{Abwassertechnik}}\\
Bypass&Absperrklappe&Umleitung für Wartungsarbeiten und Störungen an den Wasserräder&&43\\
&&&&\\
\hline
\end{tabular}
\end{table}

Im Grobkonzept 1 sollen 43 Wasserräder direkt in die Fallleitung eingebaut werden. Mit jeweils einem Generator pro Wasserrad wird Strom erzeugt. Damit der Strom der einzelnen Wasserräder zusammengeführt werden können. Muss der Wechselstrom zuerst in Gleichstrom umgewandelt werden. Dieser wird auf einen DC-Bus gelegt und anschliessend mit einem Wechselrichter auf Netz-Spannung umgewandelt. Ein Kontrollsystem überwacht die Energiegewinnung und schreitet bei Störungen ein. In unserem Hochhausmodell (Park Avenue 432) wird immer nach zwei Etagen ein Wasserrad eingebaut, um die maximale Leistung herausholen zu können. 

\bigskip

\textbf{Vorteile:}								\newline
+	kleiner Umbau der vorhandenen Anlage			\newline						
+	einfach										\newline
	
\textbf{Nachteile:}								\newline
- 	defekt anfällig								\newline
-	unregelmässige Wassermenge						\newline
-	kleine Leistung								\newline

