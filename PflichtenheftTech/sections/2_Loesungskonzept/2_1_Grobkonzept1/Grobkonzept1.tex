\subsection{Grobkonzept 1} \label{subsec:grobkonzept1}
\definecolor{dpink}{HTML}{FF13F8}
\definecolor{hpink}{HTML}{FF98FC}
\definecolor{hgruen}{HTML}{9EFFA9}
\definecolor{hblau}{HTML}{51D1FF}
\definecolor{dblau}{HTML}{008BBD}
\definecolor{hgelb}{HTML}{FFFC9E}
\definecolor{dgelb}{HTML}{FFCC67}
\definecolor{dgruen}{HTML}{00AC14}

\newcommand{\titleCell}[2]{\multicolumn{3}{c}{\cellcolor{#1}#2}}
\newcommand{\cC}[1]{\cellcolor{#1}}

\begin{table}[H]
\begin{tabular}{>{\columncolor{hgelb}}l>{\columncolor{dgelb}}l>{\columncolor{hgelb}}llllll>{\columncolor{hgruen}}l>{\columncolor{dgruen}}l>{\columncolor{hgruen}}ll}

%GELBE KOLONNE						BLAUE UND PINKIGE KOLONNE										GRUENE KOLONNE
\titleCell{hgelb}{\textbf{Turbine}}	&&\titleCell{hblau}{\textbf{Elektrotechnik}}						&&\titleCell{hgruen}{\textbf{Abwassertechnik}}&\\
&\textbf{Turbinentyp}					&&&\cC{hblau}	&\cC{dblau}\textbf{Wechselrichter}			&\cC{hblau}	&&&\textbf{Tanks}				&&\\
&Pelton-Turbine							&&&\cC{hblau}	&\cC{dblau}\textbf{Speicherakkus}			&\cC{hblau}	&&&\textbf{Bypass}				&&\\
&\textbf{Generatortyp}					&&&\cC{hblau}	&\cC{dblau}									&\cC{hblau}	&&&	für Turbine									&&\\
&Gleichstrom								&&&\titleCell{hblau}{ }																						&&&								&&\\
&\textbf{Anzahl}							&&&&													&															&&&								&&\\
%ACHTUNG WEISSER ZWISCHENRAUM
&28												&&&\titleCell{hpink}{\textbf{Bedienung}}															&&&								&&\\
&													&&&\cC{hpink}	&\cC{dpink}\textbf{Anzeige}			&\cC{hpink}							&&&\textbf{}						&&\\
&													&&&\cC{hpink}	&\cC{dpink}der Aktuellen Leistung	&\cC{hpink}						&&& 							 	&&\\
&													&&&\cC{hpink}	&\cC{dpink}								&\cC{hpink}								&&&								&&\\
&													&&&\cC{hpink}	&\cC{dpink}\textbf{}							&\cC{hpink}						&&&								&&\\
&													&&&\cC{hpink}	&\cC{dpink}							&\cC{hpink}									&&& 								&&\\
\titleCell{hgelb}{ }				&&\titleCell{hpink}{ }											&&\titleCell{hgruen}{ }&
\end{tabular}
\end{table}


Im Grobkonzept 1 soll die Pelton Turbine direkt in die Fallleitung eingebaut werden. Die Turbine wird so in das vorhandene System integriert ohne größere Umbauten zu erhalten.
In unserem Hochhausmodell (Park Avenue 432) würde immer nach zwei Etagen eine Turbine eingebaut werden. Um die maximale Leistung herausholen zu können. Die Turbine würde dort eingebaut, wo zur Sicherheitsmaßnahme die zwei 45° Bögen installiert werden. 
Die gewonnene Energie wird mittels Solaranlage in das 230V Netz transformiert und zurück in das Netz gespiesen. Da in der Solaranlage schon ein Wechselrichter vorhanden ist, wäre das die günstigste Option. Eine andere Möglichkeit besteht, mit der gewonnene Energie die Stromspeicher zu laden, um danach für gewisse Anlagen im Hochhaus direkt nutzbar zu sein.
Für die sporadische Menge vom Abwasser in der Fallleitung wird das Laufrad der Turbine mit einem Duplex-Stahl verwendet das gegen Abrasion und giftigen Gasen ist.Bei Wartungsarbeiten kann mittels Bypass an den Turbinen gearbeitet werden.

\textbf{Vorteile:}												\newline
+	kleiner Umbau der vorhandenen Anlage			\newline
+	Kostengünstiger											\newline
	
\textbf{Nachteile:}												\newline
- 	extra Beschichtung vom Laufrad					\newline
-	sporadische Wassermenge							\newline
-	Leistung														\newline


