\section{Wirtschaftlichkeit} \label{sec:wirtschaftlichkeit}
In diesem Abschnitt berechnen wir anhand grober Schätzungen die Wirtschaftlichkeit unseres Systems. 
Als Indikator für die Wirtschaftlichkeit wird die Amortisationszeit verwendet.
\subsection{Annahmen}
Für die vereinfachung der Amortisationsrechnung wurden folgende Annahmen getroffen:\\
\begin{itemize}  
\item Das Hochhaus wird neu gebaut, deshalb entstehen keine Umbaukosten.
\item Als Investitionskosten zählen wir die Einbaukosten der zusätzlichen Infrastruktur, die Materialkosten und die Entwicklungskosten.
\item Pro Jahr entsteht ein Serviceaufwand in der Höhe von schätzungsweise 3000 CHF.
\item Die Einbaukosten der zusätzlichen Infrastruktur betragen etwa 10000 CHF.
\end{itemize}

\renewcommand\arraystretch{1.2}
\newcolumntype{Z}[1]{>{\HY\RaggedLeft\bfseries}p{#1}}
\newcommand\T{\rule{0pt}{3ex}}       % Top strut
\newcommand\B{\rule[-2.5ex]{0pt}{0pt}} % Bottom strut
\subsection{Amortisationszeit}
Amortsierungszeit = Investitionskosten (Einbaukosten plus Materialkosten plus Entwicklungskosten) geteilt durch (jährlicher Rückfluss = (Stromgewinn minus Serviceaufwand)\\
\begin{table}[H]
\caption{Investitionskosten}
\small
\begin{tabular}{L{5cm}R{2cm}R{4cm}Z{3cm}}
%\multicolumn{4}{l}{\textbf{Investitionskosten}}\\
\hline
Kostenkategorie&Anzahl&Preis[CHF]/Element&Preis[CHF]\\
\hline
%\rowcolor{grau}
\multicolumn{4}{l}{\textbf{Einbaukosten}}\T\\
Einbau zus. Infrastruktur&1&10'000&10'000\B\\
%\rowcolor{grau}
\multicolumn{4}{l}{\textbf{Materialkosten}}\T\\
\multicolumn{4}{l}{\textit{Mechanik}}\\
Rohrkette 60.08m&5&10'000&50'000\\
Rohrkette 80.24m&1&13'000&13'000\\
Zahnradsystem&6&1000&6'000\\
\multicolumn{4}{l}{\textit{Elektrotechnik}}\T\\
Generator&6&110&660\\
Gleichrichter&6&300&1'800\\
Wechselrichter&1&3'401&3'401\\
Kontrollsystem&1&5900&5900\\
\multicolumn{4}{l}{\textit{Abwassertechnik}}\T\\
Umleitventil&74&???&2'082\\
Absperrklappe&74&???&\B\\
%\rowcolor{grau}
\multicolumn{4}{l}{\textbf{Entwicklungskosten}}\T\\
Software&1&48'000&48'000\B\\
\hline
%\rowcolor{grau}
\multicolumn{3}{l}{\textbf{Total}}&???\T\\
&&&\\
&&&\\
\end{tabular}
\end{table}