\section{Wirtschaftlichkeit} \label{sec:wirtschaftlichkeit}
In diesem Abschnitt berechnen wir anhand grober Schätzungen die Wirtschaftlichkeit unseres Systems. 
Als Indikator für die Wirtschaftlichkeit wird die Amortisationszeit verwendet.
\subsection{Annahmen}
Für die vereinfachung der Amortisationsrechnung wurden folgende Annahmen getroffen:\\
\begin{itemize}  
\item Das Hochhaus wird neu gebaut, deshalb entstehen keine Umbaukosten.
\item Als Investitionskosten zählen wir die Einbaukosten der zusätzlichen Infrastruktur, die Materialkosten und die Entwicklungskosten.
\item Pro Jahr entsteht ein Service- und Reperaturaufwand in der Höhe von schätzungsweise 1'000 CHF.
\item Die Einbaukosten der zusätzlichen Infrastruktur betragen etwa 10'000 CHF.
\item Es steht keine Wohnung leer.
\item Der Wasserverbrauch ist konstant.
\end{itemize}

\renewcommand\arraystretch{1.2}
\newcolumntype{Z}[1]{>{\HY\RaggedLeft\bfseries}p{#1}}
\subsection{Kosten}
\begin{table}[H]
\small
\begin{tabular}{L{4cm}R{1.5cm}rR{2cm}Z{2cm}}
%\multicolumn{4}{l}{\textbf{Investitionskosten}}\\
\hline
\textbf{Kostenkategorie}&\textbf{Anzahl}&\textbf{Preis[CHF]/Element}&\textbf{Preis[CHF]}&\\
\hline
\rowcolor{hellgrau}
\multicolumn{4}{l}{\textbf{Einbaukosten}}&10'000\T\\
Einbau zus. Infrastruktur&1&10'000&10'000&\B\\
\rowcolor{hellgrau}
\multicolumn{4}{l}{\textbf{Materialkosten}}&88'160\T\\
\multicolumn{4}{l}{\textit{Mechanik}}&\normalfont{69'000}\\
Rohrkette 60.08m&5&10'000&50'000&\\
Rohrkette 80.24m&1&13'000&13'000&\\
Getriebe&6&1000&6'000&\\
\multicolumn{4}{l}{\textit{Elektrotechnik}}&\normalfont{11'760}\T\\
Generator&6&110&660&\\
Gleichrichter&6&300&1'800&\\
Wechselrichter&1&3'401&3'401&\\
Kontrollsystem&1&5900&5900&\\
\multicolumn{4}{l}{\textit{Abwassertechnik}}&\normalfont{7'400}\T\\
Umleitventil&74&100&7'400&\B\\
\rowcolor{hellgrau}
\multicolumn{4}{l}{\textbf{Entwicklungskosten}}&48'000\T\\
Software&1&48'000&48'000\B\\
\hline
%\rowcolor{grau}
\multicolumn{3}{l}{\textbf{Gesamt}}&&146'160\T\\
&&&\\
&&&\\
\end{tabular}
\caption{Kostentabelle}\label{tab:kostentabelle}
\end{table}
\newpage
\subsection{Amortisationszeit}
Mit der Formel $A = \tfrac{K}{R}$ wird nun die Amortisationszeit (A) berechnet, wobei K die einmaligen Investitionskosten sind, die man aus der Tabelle \ref{tab:kostentabelle} entnehmen kann. R ist der jährliche Rückfluss, also in unserem Fall der Ertrag aus dem Stromgewinn abzüglich des Service- und Reperaturaufwands.\\

\bigskip
K = 146'160 CHF

\bigskip
R = 365$\tfrac{T}{J}\cdot$10.62$\tfrac{CHF}{T}$ - 1'000$\tfrac{CHF}{J}$ = 2'876$\tfrac{CHF}{J}$

\bigskip
$A = \tfrac{146'160CHF}{2'876 \tfrac{CHF}{J}} =$ 50.82$J$

\bigskip
Trotz Verwendung optimistisch geschätzter Werte braucht es 51 Jahre, bis die Investitionskosten amortisiert sind. In der Regel sind z.B. Solaranlagen nach 9-15 Jahren amortisiert bei einer Lebensdauer von 30 Jahren \cite{helion}. Für unser System wären 10-15 Jahre Amortisationszeit wünschenswert gewesen, aber nicht realisierbar.
Angenommen das System würde ein zweites Mal eingebaut, könnten die Entwicklungskosten für die Software, also 48'000 CHF, gespart werden. Aber auch dann wäre das System erst nach 30 Jahren amortisiert. Unser System ist also unwirtschaftlich.