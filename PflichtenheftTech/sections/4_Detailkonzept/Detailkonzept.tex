\section{Detailkonzept} \label{sec:detailkonzept}

Das Konzept mit den Wasserliften ist am besten geeignet für unsere Anwendung. Es existieren bereits solche «Rohrkettenförderer», die jedoch Produkte hinaufbefördern. Wir nutzen dieses System um das Wasser nach unten zu befördern und dabei Energie zu gewinnen. Es werden insgesamt sechs Lifte benötigt. Fünf Lifte überwinden je 60.08\si{m} und der unterste Lift überwindet 80.24\si{m}. In der Abbildung \ref{fig:PrinzipGrobkonzept4} \nameref{fig:PrinzipGrobkonzept4} ist dies grafisch dargestellt.

\subsection{Elektronik}

\paragraph{Generator}
\begin{figure}[H]
\centering
\includegraphics[width=\linewidth]{tagesGangKurve.png}
\caption{Typische Tagesgangkurve. \cite{peakWaterDemand}}
\label{fig:tagesGangKurve}
\end{figure}
\paragraph{Anzeige}

\newpage
\paragraph{Wechselrichter}
Damit die gewonnen Leistung von dem Asynchrongenerator in das Netz eingespiesen werden kann, muss der Wechselrichter folgende Eigenschaften aufweisen. 

Leistung: 	9KW \newline
Ausgang:	3Phasen \newline
Kosten:		Die Kosten sollen möglichst gering gehalten werden. \newline

In der Förderungsanlage wird ein Asynchrongenerator verbaut. Die Firma Voltacon ist bekannt für ihre Hochleistungswechselrichter.

 Das Model Hybrid Wechselrichter HSI10000 entspricht den gewünschten Anforderungen für unsere Förderungsanlage. Der Generator lieferte eine gewisse Spannung die vom Gleichrichter auf die 48VDC geglättet wird. Der Wechselrichter transformiert die 48VDC in 230VAC mit einer Frequenz von 50 Hertz. Das Gerät kann bis zu einer Leistung von 10KW erbringen. Mittels integrierten Displays kann die die erbrachte Leistung abgelesen werden. 


\cite{•}

\subsection{Mechanik}

\paragraph{Rohrkette}

In der Industrie werden Rohrkettenförderer für den Transport von Schuttgüter verwendet. In der Abbildung \ref{fig:Rohrkettenfoerderer} \nameref{fig:Rohrkettenfoerderer} ist der Innenaufbau eines solchen Rohrkettenförderers ersichtlich.

\begin{figure} [H]
	\centering
	\includegraphics[width=6cm]{Rohrkettenfoerderer.jpg}
	\caption{Innenaufbau Rohrkettenförderer \cite{abconvey}}
	\label{fig:Rohrkettenfoerderer}
\end{figure}

Wir wollen keinen Schutt nach oben befördern, daher muss dieses System auf unsere Anforderungen angepasst werden. Diese Anforderungen sind, dass die verwendeten Materialien Robust gegenüber Korrasion sind, da das Abwasser aggressiv auf diese wirkt. Weiter müssen, um einen möglichst hohen Wirkungsgrad zu erreichen, die Platten mit möglichst kleinem Spielraum zur Ausserwand konstruiert werden, damit das Wasser nicht einfach auf der Seite herunterfliessen kann und gleichzeitig nicht eine zu grosse Reibung erzeugt wird. Die Drehachse, an dem der Generator angeschlossen wird ist ein Stösselkettenrad. Dieser ist in der Abbildung \ref{fig:stoesselkettenrad} \nameref{fig:stoesselkettenrad} zu sehen

\begin{figure} [H]
	\centering
	\includegraphics[width=6cm]{Stoesselkettenrad.jpg}
	\caption{Stösselkettenrad \cite{schrage}}
	\label{fig:stoesselkettenrad}
\end{figure}


Um diesen Wasserlift zu bauen beauftragen wir die Firma Schrage, ein führender Speziallist für Rohrketten, die in Deutschland zu Hause ist, beauftragt. Die Kosten belaufen sich für die 60.08\si{m} Höhendifferenz auf ca. 10'000\si{Fr} pro Lift und für die 80.24\si{m} Höhendifferenz auf ca. 13'000\si{Fr}. Insgesamt würde die Analage mit den Rohrketten und Stösselkettenrad insgesamt ca.63'000\si{Fr} kosten. \cite{schrage}

\paragraph{Rohr}

Gerberit...

\newpage

\subsection{Kosten}