\section{Reflexion}
\par Mit dieser Reflexion wird die Projektarbeit kritisch beurteilt, so dass wichtige Schlüsse gezogen und die Qualität der noch folgenden Projekte verbessert werden kann. Nachfolgend wird auf die negativen und positiven Punkte der Arbeit näher eingegangen.

\subsection{Gesamtreflexion negativ}
\par Grosses Verbesserungspotenzial liegt in der Projektleitung, die zum Teil stark vernachlässigt wurde. Dies ist auf eine ungünstige Verteilung der Arbeitspackete zurückzuführen. Der Projektleiter hatte zu viele Arbeitspakete an sich selbst verteilt, so dass ihm für die wichtige Projektsteuerung nicht genügend Zeit blieb. An dieser Stelle hätte der stellvertretende Projektleiter eingreifen müssen, aber auch er hatte mit seinen Aufgaben zu viel Arbeit. Zukünftig muss der Projektleiter mehr entlastet werden, damit er sich besser auf die Projektsteuerung konzentrieren kann.
\par Einzelnen Teammitgliedern fehlte es an Motivation und Arbeitslust. Obwohl Regeln für die rechtzeitige Abgabe und Qualität der Arbeitspakete definiert worden sind, wurden sie auch nach mehreren Aufforderungen nicht eingehalten. Solches Verhalten soll in folgenden Projekten nicht mehr toleriert werden. Um die Qualität sicherzustellen, hätte man die Teammitglieder klar auf Qualitätsmängel anspechen und Konsequenzen kommunizieren sollen. Stattdessen übernahmen andere Teammitglieder die Arbeit, so dass sich das Ungleichgewicht weiter verschärfte. 
\par Die Sitzungen waren nicht effizient genug. Sie hatten oft keinen Ablauf und endeten mit unnötigen und langen Diskussionen. Ein Grund dafür war die schlechte Vorbereitung des Sitzungsleiters. Damit jedes Teammitglied lernt, Sitzungen zu leiten, wechselte die Sitzungsleitung bei jeder Sitzung. Dies hat jedoch noch mehr zur Inneffizienz beigetragen. Zukünftig wird der Projektleiter verantwortlich für die Sitzungen sein und diese auch angemessen vorbereiten.
\par Nach der Planung und der Ausarbeitung der Konzepte sollten keine neuen Ideen mehr aufgenommen werden. Nachdem drei verschiedene Konzepte geplant und ausgearbeitet wurden, wurde relativ spät ein neues Konzept vorgeschlagen und vom Team angenommen. Die Eingliederung dieses Konzepts führte zu einem merklichen Mehraufwand.

\subsection{Gesamtreflexion positiv}
\par Ein sehr positiver Aspekt war die Arbeit mit LaTex und GitHub. Anfänglich musste viel Zeit für die Einarbeitung investiert werden, aber später überwiegten die Vorteile. Der grosse Vorteil von LaTex ist, dass das Dokument in einzelne Files unterteilt werden kann. So konnten alle Teammitglieder gleichzeitig am gleichen Dokument arbeiten. Zudem hat das Tool jederzeit die optimale und einheitliche Darstellung gewährleistet und Formfehler verhindert. Mit der GitHub-Datenbank konnte von überall auf den aktuellsten Stand des Dokuments zugegriffen werden. Alle Änderungen am Dokument konnten durch das Tool dokumentiert, gespeichert, nachvollzogen und auch wieder rückgängig gemacht werden. Dem Projektleiter war es dadurch möglich zu überprüfen, wer was getan hat und ob die Arbeiten rechtzeitig hochgeladen wurden. Diese beiden Tools haben die Zusammenarbeit verbessert und werden deshalb auch in Zukunft verwendet.
\par Ein weiterer positiver Punkt war der Wechsel der Projektleitung. Der Erste Projektleiter brach das Studium bereits nach wenigen Wochen ab. Schnell und unkompliziert konnte aber ein neuer Projektleiter bestimmt werden. Solche Situationen sind ein Risiko für das Projekt, daher ist es wichtig, dass auch in folgenden Projekten auch immer ein Stellvertreter bestimmt wird.
