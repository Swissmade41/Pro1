\documentclass[12pt]{article}
\title{Reflexion}

\begin{document}

\section{Reflexion}

Mit dieser Reflexion wollen wir unsere Projektarbeit kritisch beurteilen, um wichtige Schlüsse aus der Arbeit ziehen zu können und somit die Qualität der noch folgenden Projekte zu verbessern.  Nachfolgend wird auf den negativen und positiven Punkten unsere Arbeit näher eingegangen.
\bigskip

\subsection{Gesamtreflexion negativ}

Ein wichtiger Punkt in dem wir noch Verbesserungspotenzial haben ist die Projektleitung. Diese wurde zum Teil stark vernachlässigt. Dies ist auf die Verteilung der Arbeitspackete zurückzuführen. So hat der Projektleiter zu Beginn sich selber zu viel Arbeit auf gebürgt und hatte für das Steuern des Projektes keine Zeit mehr. An dieser Stelle hätte der stellvertretende Projektleiter eingreifen müssen, aber auch er hatte mit seinen Aufgaben zu viel Arbeit. Zukünftig muss der Projektleiter mehr entlastet werden, damit er sich besser auf die Steuerung des Projektes konzentrieren kann. Ein weiterer Punkt war die fehlende Motivation und Arbeitslust einzelner Teammitgliedern. Obwohl wir Regeln für die rechtzeitige Abgabe und Qualität der Arbeitspackete definiert haben, wurden diese Regeln auch nach mehreren Aufforderungen nicht eingehalten. Solches Verhalten werden wir für die nächsten Arbeiten nicht mehr tolerieren. Anstatt direkt zu handeln und die Konsequenzen für diejenigen Personen aufzuzeigen, übernahmen andere die Arbeit, um sicherzustellen, dass wir am Ende auch eine qualitativ gute Arbeit abgeben können. Dies Alles führte zu einem grossen Ungleichgewicht der Arbeitsstunden von den einzelnen Personen. 
\\
\\
Ein weiterer Kritikpunkt ist, dass die Sitzungen effizienter werden müssen. Diese hatten oft keinen Ablauf und endeten mit unnötigen und langen Diskussionen. Ein Grund dafür war die schlechte Vorbereitung des Sitzungsleiters auf die Sitzungen. Dieser wurde zusätzlich nach jeder Sitzung gewechselt, damit jeder diese Erfahrung machen kann. Dies hat jedoch noch mehr zur Inneffizienz beigetragen. Zukünftig wird der Projektleiter verantwortlich für die Sitzungen sein und diese auch angemessen vorbereiten.
\\
\\
Der letzte Punkt ist, dass nach der Planung und der Ausarbeitung der Konzepte keine neuen Ideen mehr aufgenommen werden. Nachdem wir 3 verschiedene Konzepte geplant und ausgearbeitet haben, kam uns eine viel bessere Möglichkeit in den Sinn unser Problem zu lösen. Die Eingliederung dieses Konzepts führte zu einem merklichen Mehraufwand. In der Zukunft werden wir auf die zu Beginn festgelegte Konzepte beschränken.
\bigskip


\subsection{Gesamtreflexion positiv}

Ein sehr positiver Aspekt war die Arbeit mit LaTex und GitHub.  Wir mussten zu Beginn des Projektes zwar mehr Zeit investieren um die Tools bedienen zu können, aber konnten damit schneller arbeiten. Der grosse Vorteil an LaTex ist, dass das Dokument in einzelne Files unterteilt werden kann. Alle konnten so ohne grössere Probleme gleichzeitig am gleichen Dokument arbeiten, indem nur für das Arbeitspacket zuständige File bearbeitet wurde.  Zudem hat das Tool jederzeit die optimale und einheitliche Darstellung gewährleitet und verhindert somit Formfehler. Mit der GitHub Datenbank hatten alle immer Zugriff auf den aktuellsten Stand. Alle Änderungen am Dokument konnten durch das Tool dokumentiert und gespeichert werden. Jede Änderung konnte so nachvollzogen und auch wieder rückgängig gemacht werden. Dem Projektleiter war es auch mit dem Tool möglich zu überprüfen wer was getan hat und ob die Arbeiten rechtzeitig hochgeladen wurden. Diese beiden Tools haben unsere Zusammenarbeit verbessert und daher werden wir auch in Zukunft mit diesen Mitteln arbeiten.
\\
\\
Ein weiterer positiver Punkt war der Wechsel der Projektleitung. Der Erste Projektleiter brach das Studium bereits nach wenigen Wochen ab. Wir konnten uns aber schnell der neuen Situation anpassen und einen neuen Projektleiter bestimmen. Solche Situationen sind immer ein Risiko in einem Projekt, daher ist es wichtig das wir auch in den nächsten Projekten immer einen Stellvertreter bestimmen.

\end{document}
