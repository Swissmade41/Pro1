\documentclass[12pt]{article}
\title{Management Summary}

\begin{document}

\section*{Management Summary}
%------Problemstellung
%Worum ging es? Welches Problem sollte gelöst werden?
%(FRANK) 
In Hochhäusern und Wolkenkratzern fliesst Abwasser von hoch oben nach unten. Die potentielle Energie, die das Wasser in den oberen Stockwerken hat, bleibt ungenutzt. Das Ziel der Gruppe 4 war, herauszufinden, ob und wie man diese potentielle Energie am besten nutzen könnte.\\
%(MICHEL)


%------Vorgehen
%Wie sind Sie vorgegangen, um die Aufgabe zu lösen?
%(FRANK) 
Von Beginn an hatte die Gruppe 4 den Lösungsansatz, mit einer Turbine im Abwasserrohr die Energie zu nutzen. Bei der Rechrche hat sich gezeigt, dass der Luftwiderstand innerhalb des Rohres ein Problem ist. Dazu wurden vier verschiedene Lösungsansätze entwickelt. Wie viel Energie die diese jeweils liefern könnten, berechnete die Gruppe 4 an einem Hochhausmodell. Durch eine Nutzwertanalyse wurden die Lösungen miteinander verglichen und entschieden, welche weiter verfeinert werden soll.\\
%(MICHEL)


%------Hauptergebnisse
%Was sind die wesentlichen Ergebnisse – positive und negative Abweichungen?
Das entstandene Detailkonzept beschreibt eine effiziente Lösung, sie ist jedoch nicht sehr Wirtschaftlich. 
%(RONI)
%(LARS)

%------Handlungsempfehlung
%In welcher Weise können die Ergebnisse verwendet werden? Welche Probleme bleiben ungelöst?
%(PASCAL)
%(LARS)
\end{document}
