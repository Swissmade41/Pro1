\documentclass[12pt]{article}
\title{Management Summary}

\begin{document}

\section*{Management Summary}
%------Problemstellung
%Worum ging es? Welches Problem sollte gelöst werden?
%(FRANK) 
In Hochhäusern und Wolkenkratzern fliesst Abwasser von hoch oben nach unten. Die potentielle Energie, die das Wasser in den oberen Stockwerken hat, bleibt ungenutzt. Das Ziel der Gruppe 4 war, herauszufinden, ob und wie man diese potentielle Energie am besten nutzen könnte.\\
%(MICHEL)


%------Vorgehen
%Wie sind Sie vorgegangen, um die Aufgabe zu lösen?
%(FRANK & LARS) 
Von Beginn an hatte die Gruppe 4 den Lösungsansatz, mit einer Turbine im Abwasserrohr die Energie zu nutzen. Bei der Recherche hat sich gezeigt, dass der Luftwiderstand innerhalb des Abwasserrohres einen deutlich negativen Effekt auf die Energiegewinnung hat. Deshalb wurden zwei Grobkonzepte erarbeitet, bei denen der Luftwiderstand durch Verwendung von Tanks und Druckleitungen minimiert wird und zum Vergleich ein Grobkonzept, bei dem nichts gegen den Luftwiderstand unternommen wurde. Schlussendlich ist noch ein viertes Grobkonzept entstanden, das nicht mit Turbine, sondern mit einem Wasserlift funktioniert. Wie viel Energie die jeweiligen Grobkonzepte liefern könnten, berechnete die Gruppe 4 an einem Hochhausmodell. Durch eine Nutzwertanalyse wurden die Lösungen miteinander verglichen und entschieden, dass das Grobkonzept mit dem Wasserlift weiter ausgearbeitet werden soll.\\


%------Hauptergebnisse
%Was sind die wesentlichen Ergebnisse – positive und negative Abweichungen?
%(LARS)
Das entstandene Detailkonzept beschreibt eine effiziente Lösung, sie ist jedoch nicht sehr wirtschaftlich. 
%(RONI)
Das Grobkonzept zeigt in hinsicht der Technik eine Möglichkeiten die Abwasserenergie optimal auszunutzen. Mit den Berechnungenn kamm raus, dass das Grobkonzept mit dem Wasserlift am meisten Energie erzeugt.Um grössere Wartungsarbeiten zu ermöglichen wurden Ventile eingebaut, die das Abwasser in eine zusätzlich Fallleitung ablenkt.Die Kosten für den Bau und die Elektronik in diesem Grobkonzept sind extrem hoch, wie das Detailkonzept aufzeigt und somit wird eine längere Amortisationszeit enstehen. So wird das Projekt in finanzieller Hinsicht sich nicht rentieren. 
%(FRANK)
Es zeigte sich, dass das Grobkonzept mit Wasserlift die potentielle Energie des Abwassers am effizientesten umsetzt. Aber um Reperaturen am System zu ermöglichen, müssen Ventile im Abwasserrohr eingebaut werden, die von einem Computer aus angesteuert werden können. Dazu braucht es eine Software, deren Entwicklungskosten hoch sind und somit die Amortisationszeit verschlechtern. Die teuren Wasserlifte mit den ebenso schwer ins Gewicht fallenden Reperaturkosten verschlechtern die Amortisationszeit noch weiter, so dass eine Umsetzung undenkbar wird.


%------Handlungsempfehlung
%In welcher Weise können die Ergebnisse verwendet werden? Welche Probleme bleiben ungelöst?
%(FRANK)
Die Normen sind nie berücksichtigt worden. Falls also ein neues, wirtschaftliches Konzept gefunden wird, muss geprüft werden, ob es mit den gesetzlich geregelten Normen übereinstimmt. Eine weitere Schwierigkeit ist die Verstopfungsgefahr. Es ist nie überprüft worden, wie anfällig auf Verstopfung das System wirklich ist. Verstopfungen und hohe Wartungskosten müssen verhindert werden können. Die bisherigen Ergebnisse zeigen also, dass einer Realisierung eines solchen Vorhabens einige Steine im Weg stehen.
%(MICHEL)
\end{document}
