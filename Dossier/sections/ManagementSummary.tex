\documentclass[12pt]{article}
\title{Management Summary}

\begin{document}

\section*{Management Summary}
%------Problemstellung
%Worum ging es? Welches Problem sollte gelöst werden?
%(FRANK) 
In Hochhäusern und Wolkenkratzern fliesst Abwasser von hoch oben nach unten. Die potentielle Energie, die das Wasser in den oberen Stockwerken hat, bleibt ungenutzt. Das Ziel der Gruppe 4 war, herauszufinden, ob und wie man diese potentielle Energie am besten nutzen könnte.\\
%(MICHEL)


%------Vorgehen
%Wie sind Sie vorgegangen, um die Aufgabe zu lösen?
%(FRANK & LARS) 
Von Beginn an hatte die Gruppe 4 den Lösungsansatz, mit einer Turbine im Abwasserrohr die Energie zu nutzen. Bei der Recherche hat sich gezeigt, dass der Luftwiderstand innerhalb des Abwasserrohres einen deutlich negativen Effekt auf die Energiegewinnung hat. Deshalb wurden zwei Grobkonzepte erarbeitet, bei denen der Luftwiderstand durch Verwendung von Tanks und Druckleitungen minimiert wird und zum Vergleich ein Grobkonzept, bei dem nichts gegen den Luftwiderstand unternommen wurde. Schlussendlich ist noch ein viertes Grobkonzept entstanden, das nicht mit Turbine, sondern mit einem Wasserlift funktioniert. Wie viel Energie die jeweiligen Grobkonzepte liefern könnten, berechnete die Gruppe 4 an einem Hochhausmodell. Durch eine Nutzwertanalyse wurden die Lösungen miteinander verglichen und entschieden, welche weiter ausgearbeitet werden soll.\\
%(RONI)

%------Hauptergebnisse
%Was sind die wesentlichen Ergebnisse – positive und negative Abweichungen?
%(LARS)
Das entstandene Detailkonzept beschreibt eine effiziente Lösung, sie ist jedoch nicht sehr wirtschaftlich. 
%(RONI)
%(FRANK)
Es zeigte sich, dass das Grobkonzept


%------Handlungsempfehlung
%In welcher Weise können die Ergebnisse verwendet werden? Welche Probleme bleiben ungelöst?
%(FRANK)
Die Ergebnisse 
%(MICHEL)
\end{document}
