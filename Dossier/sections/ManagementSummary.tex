\section*{Management Summary}
\par In Hochhäusern und Wolkenkratzern fliesst Abwasser aus grossen Höhen zurück in die Kanalisation. Die potentielle Energie, die dabei umgesetzt wird, bleibt gänzlich ungenutzt. Das Ziel war, herauszufinden, ob und wie diese Energie am besten genutzt werden könnte.\smallskip
\par Von Beginn an bestand der Lösungansatz darin, die kinetische Energie des Abwssers mit einer Turbine in elektrische Energie umzusetzen. Die Recherche hat ergeben, dass der Luftwiderstand, den das Abwasser erfährt, die Fliessgeschwindigkeit und somit die mögliche Energiemenge, die gewonnen werden kann, stark begrenzt. Aus diesem Grund wurden zwei Grobkonzepte erarbeitet, bei denen der Luftwiderstand durch Verwendung von Tanks und Druckleitungen minimiert wird, und zum Vergleich ein Grobkonzept, bei dem nichts gegen den Luftwiderstand unternommen wurde. Letztendlich entstand auch ein viertes Grobkonzept, bei welchem nicht eine Turbine, sondern ein Wasserlift eingesetzt wird. Die mögliche Energiemenge, die durch die jeweiligen Grobkonzepte gewonnen werden kann, wurde anhand eines Hochhausmodells berechnet. In einer Nutzwertanalyse wurden die Lösungsansätze miteinander verglichen und entschieden, dass das Grobkonzept mit dem Wasserlift weiter untersucht werden soll.\smallskip
\par Es hat sich herausgestellt, dass das Grobkonzept mit dem Wasserlift den höchsten Wirkungsgrad erzielt. Damit Wartungsarbeiten durchgeführt werden können, muss das Abwasser durch einen Computer über Ventile in zusätzliche Fallleitungen umgelenkt werden können, was die Kosten für das System erhöht. Die teuren Wasserlifte mit ebenfalls kostenintensiven Wartungsarbeiten verlängern die Amortisationszeit, so dass eine Umsetzung undenkbar wird.\smallskip
\par Geltende Industrienormen und gesetzliche Grundlagen sind nicht berücksichtigt worden. Sollte ein neues, wirtschaftlicheres Konzept gefunden werden, muss überprüft werden, ob es mit letzteren vereinbar ist. Es wurde nicht untersucht, wie resistent das System gegenüber Verstopfungen durch Festkörper im Abwasser ist. Verstopfungen und hohe Reparaturkosten müssen deshalb vermieden werden. Die bisherigen Ergebnisse zeigen, dass der Realisierung eines solchen Vorhabens einige Steine im Weg liegen.\smallskip