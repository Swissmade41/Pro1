\section{Einleitung}

\subsection{Ausgangslage}

Weltweit wachsen Städte immer mehr in die Höhe. Um in hohen Gebäuden Trinkwasser in die oberen Stockwerke zu pumpen, wird viel Energie benötigt. Das entstehende Abwasser hat eine dementsprechend hohe potentielle Energie, die ungenutzt bleibt, wenn das Wasser zurück in die Kanalisation fliest. Zudem muss das Wasser meistens noch abgebremst werden, bevor es zurück in die Kanalisation geleitet werden kann. Dabei geht die Energie in Form von Wärme verloren. 

Um Energie zurück zu gewinnen, soll das Abwasser durch eine Turbine geführt werden, die einen Generator antreibt. Damit kann der Strom zurück zu den Wasserpumpen geführt werden, die frisches Trinkwasser in die oberen Stockwerke pumpen. Alternativ kann der Strom auch in das Stromnetz zurückgespeist werden. 

Im Rahmen des Pro1E wollen wir ein solches Abwasser - Kleinkraftwerk unter den Aspekten der Machbarkeit, Wirtschaftlichkeit und des Umweltschutzes untersuchen.  

\subsection{Ziel des Dokuments}

Das Ziel dieses Dokumentes ist es, die Resultate der Recherche zu unserm Produkt aufzuzeigen. Dabei wollen wir herausfinden, ob das Abwasser - Kleinkraftwerk genügend Energie zurückgewinnen kann, wie die Sicherheit des Geräts gewährleistet werden kann und ob bereits ähnliche Produkte auf dem Markt zu finden sind. Des Weiteren wollen wir herausfinden ob das Gerät einfach in die Infrastruktur eines Gebäudes eingebaut und in bereits bestehende Systeme integriert werden kann.

\subsection{Produktbedingungen}

Unser Abwasser – Kleinkraftwerk soll möglichst viel Energie zurückgewinnen, dies ist nur möglich durch einen hohen Wirkungsgrad und einen niedrigen Stromverbrauch des Geräts. Weiter soll das Gerät in mehreren Ausführungen mit unterschiedlichen Rohrdurchmesser erhältlich sein. Somit wird garantiert, dass es einfach in schon bestehenden Leitungen eingebaut werden kann. Das Gerät soll zudem möglichst verstopfungssicher sein. Kommt es trotzdem zu einer Verstopfung muss das Gerät einfach gereinigt werden können. Um die Energiegewinnung zu kontrollieren und Fehlermeldungen (z.B Verstopfungen) mitzuteilen, soll das gerät kommunikationsfähig sein und auch an bestehende Hausautomation-Systeme angeschlossen werden können.


