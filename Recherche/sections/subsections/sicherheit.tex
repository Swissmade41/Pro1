\subsection{Sicherheit}


\subsubsection{Recherchegrund}
Wir möchten über die Sicherheit der Abwasserrohre und die entstehenden Abgase recherchieren, um gewisse Risiken einschätzen zu können. 
\subsubsection{Ergebnisse}
Die Verstopfungsgefahr in den Abwasserrohren ist sehr klein. %es geht ja nicht nur um die Verstopfung in Rohren ohne Turbine. Es geht besonders um die Verstopfung mit Turbine. Dies scheint ein verbreitetes Problem zu sein. Vielleicht kannst du verschiedene Verstopfungsarten/Gefahren für Turbinen auflisten. Zsbp Haare oder tampons usw. Auch die Abnutzung durch Kalk usw müsste studiert werden.
Eine Verstopung passiert selten, weil das Schmutzwasser nur geringen Abständen in das Rohr hineinfliesst. Die Möglichkeit besteht, diese Verstopfungen ganz zu eliminieren, in dem man eine Hebeanlage in den Toilettenräumen einbaut. Mit diesen Hebeanlagen entsteht aber zusätzlicher Lärm. In diesen Bereichen werden flexible Rohre benützt, um Rohrbrüche zu vermeiden.
\paragraph{Leitungen}
Bei Fallleitungen die Grösser als 10 bis 20\si{m} sind, schreibt die Norm, dass man zwei 45°-Bögen mit einem geraden Stück von 250\si{mm} Länge zum Druckabbau eingebaut werden. Heutzutage kann man die Rohre so wählen, dass gar kein Rohrbruch entsteht.
\paragraph{Abgase}
Die Abgase (Schwefelgas) die entstehen, werden Mittels zusätzlicher Lüftungsleitung abgeleitet und über das Dach in die Umwelt gelassen. In den Hebeanlagen kann es durch Gärprozesse zur Gasbildung kommen. Den dadurch entstehenden Unterdruck im Raum muss man durch Lüftung selber regeln. 
\paragraph{Berechnung}
Wartungstechnisch werden Abwasserrohre alle fünfzehn Jahre auf Dichtigkeit kontrolliert. Die Rohre werden so gewählt, dass sie ein Leben lang halten.

\subsubsection{Fazit}

Die Sicherheit ist in vielen Bereichen gewährleistet. Wenn die Fallleitung einen 45° Bogen hat, wirkt sich das auf die Fliessgeschwindigkeit des Schmutzwassers und somit negativ auf die Energiegewinnung aus. Zusätzlich wird mit der Installation einer Hebeanlage Energie aus dem Netz bezogen und Lärm verursacht.%Ist also nicht gerade sehr positiv. Gibt es andere Arten Verstopfungen zu vermeiden? Zbsp optimierte Beschaufelung der Turbine? 

\clearpage 





