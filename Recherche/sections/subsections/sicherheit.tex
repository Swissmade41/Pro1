\subsection{Sicherheit}


\subsubsection{Recherchegrund}
Wir möchten die Risiken beim Einbau einer Turbine im Abwassersystem einschätzen.

\subsubsection{Ergebnisse}
Die Verstopfungsgefahr einer im Abwasserrohr eingebauten Turbine ist sehr klein. 
Eine Verstopfung zum Beispiel durch Tampons, Haare oder Feuchttücher passiert selten, weil das Schmutzwasser nur in geringen Abständen in das Rohr hineinfliesst. 
Um Verstopfungen duch Haare zu vermeiden gibt es die Einkanal-Hydrauliken (Einkanalpumpe).\cite{Homa-Pumpen}

Die Möglichkeit besteht, Verstopfungen ganz zu eliminieren, indem eine Hebeanlage in den Toilettenräumen eingebaut wird. Mit diesen Hebeanlagen entsteht aber zusätzlicher Lärm und in deren Bereichen müssen flexible Rohre installiert werden, um Rohrbrüche zu vermeiden. \cite{Hebeanlagen}\cite{TippzumBau}

\paragraph{Leitungen}
Bei Fallleitungen, die grösser als 10 bis 20 \si{m} sind, schreibt die Norm vor, dass man zwei 45°-Bögen mit einem geraden Stück von 250 \si{mm} Länge zum Druckabbau einbauen muss. Heutzutage kann man die Materialien so wählen, dass kein Rohrbruch entsteht. Bei grossem Förderstrom kann ein Bypass eingebaut werden, um die Leitungen zu entlasten. 
Ein anderes Problem ist die Ablagerung von Kalk in Leitungen. Sie entsteht in den Leitungen, die sich erhitzen und wieder abkühlen. Dabei lagert sich eine kleine Schicht Kalkstein in der Innenseite des Rohrs ab. Dies kann bei der Pelton Turbine zu Verstopfung führen.\cite{BauNetzMedia,Mediagon,ksb}

\paragraph{Abgase}
Die entstandenen Abgase (zum Beispiel Schwefelgas) werden mittels zusätzlicher Lüftungsleitung abgeleitet und über das Dach in die Umwelt gelassen. In den Hebeanlagen kann es durch Gärprozesse zur Gasbildung kommen. Der entstandene Unterdruck im Raum muss durch Lüftung selber geregelt werden. 
Ein anderes Phänomen ist, dass bei langem Transport von Fäkalien und Wasser die korrosive Schwefelwasserstoff-Säure entsteht. Zum Schutz gegen Korrosion verwenden Ingenieure den Duplex-Stahl mit Keramikbeschichtung.\cite{Water_Wastewater}



\paragraph{Wartung}
Abwasserrohre werden alle 30 Jahre auf Dichtigkeit kontrolliert. Die Materialien für die Abwasserrohre werden spezifisch gewählt, so dass sie während der Lebensdauer nie ausgewechselt werden müssen.

\subsubsection{Fazit}

Die Sicherheit der im Abwassersystem eingebauten Turbine ist gewährleistet, sofern die Normen eingehalten werden und die Verstopfungsgefahr mit einer Hebeanlage minimiert wird. 

\clearpage 





