\subsection{Sicherheit}


\subsubsection{Recherchegrund}
Wir möchten über die Sicherheit der Abwasserrohre, und über die Abgase die entstehen um gewisse Risiken einschätzen zu können. 

\subsubsection{Ergebnisse}

Die Verstopfungsgefahr in den Abwasserrohren ist sehr selten wie nie, weil das Schmutzwasser in nur geringen Abständen in das Rohr hineinfliesst. Die Möglichkeit besteht diese Verstopfungen ganz zu eliminieren in dem man eine Hebeanlage in den Toilettenräumen einbaut. Mit diesen Hebeanlagen entsteht zusätzlicher Lärm. In diesen Bereichen werden flexible Rohre benützt um Rohrbrüche zu vermeiden.

\subsubsubsection{Leitungen}
Bei Fallleitungen die Grösser als 10 bis 20 Meter schreibt die Norm das man zwei 45°-Bögen mit einem geraden Stück von 250mm Länge zum Druckabbau eingebaut werden. Heutzutage kann man die Rohre so wählen, dass gar keinen Rohrbruch entsteht.

\subsubsubsection{Abgasen}
Die Abgase (Schwefelgas) die entstehen, werden Mittels zusätzlicher Lüftungsleitung abgeleitet und über das Dach in die Umwelt gelassen. In den Hebeanlagen kann es durch Gärprozessen Gase entstehen, diesen Unterdruck muss man durch entlüften oder belüften von diesem Raum selber regeln. 

\subsubsubsection{Wartung}
Wartungstechnisch werden Abwasserrohre alle 15-Jahren auf Dichtigkeit kontrolliert. Die Rohre werden so gewählt, dass sie ein Leben lang halten.

\subsubsection{Fazit}
Die Sicherheit ist in vielen Bereichen gewährleiste, dass die Fallleitung einen 45° Bogen hat, wirkt sich auf die Fliessgeschwindigkeit des Schmutzwassers und man wird einen gewissen Verlust erhalten. Zusätzlich mit einer Installation einer Hebeanlage wird Energie aus dem Netz bezogen wie auch Lärm verursacht.

\clearpage 





