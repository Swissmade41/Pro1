\subsection{Sicherheit}


\subsubsection{Recherchegrund}
Wir möchten die Risiken bei einen Einbau einer Turbine in einem Abwassersystem  einschätzen.

\subsubsection{Ergebnisse}
DDie Verstopfungsgefahr beim Einbau einer Turbine in einer Abwasserrohren ist sehr klein. 
Eine Verstopfung zum Beispiel mit Tampons, Haaren oder Feuchttüchern passiert selten, weil das Schmutzwasser nur in geringen Abständen in das Rohr hineinfliesst. 
Für diese Verstopfungen mit Haaren gibt es heute den Einkanal-Hydrauliken der sich von diesen Materialien nicht verstopft.

Die Möglichkeit besteht, diese Verstopfungen ganz zu eliminieren, indem eine Hebeanlage in den Toilettenräumen eingebaut wird. Mit diesen Hebeanlagen entsteht aber zusätzlicher Lärm und in deren Bereichen müssen flexible Rohre installiert wird, um Rohrbrüche zu vermeiden.
\paragraph{Leitungen}
Bei Fallleitungen, die grösser als 10 bis 20\si{m} sind, schreibt die Norm, dass man zwei 45°-Bögen mit einem geraden Stück von 250\si{mm} Länge zum Druckabbau eingebaut werden. Heutzutage kann man die Materialien so wählen, dass kein Rohrbruch entsteht. Bei grossen Förderstrom kann einen Bypass eingebaut werden um die Leitungen zu entlasten. 
Ein anderes Problem ist die Ablagerung von Kalk in den Leitungen. Dies entsteht in den Leitungen die sich erhitzen und sich wieder abkühlen. Dabei lagert sich eine kleine Schicht Kalkstein in der innen Seite des Rohrs ab. Dies kann bei der Pelton Turbine zur Verstopfung führen.
\paragraph{Abgase}
DDie entstandenen Abgase (Schwefelgas), werden Mittels zusätzlicher Lüftungsleitung abgeleitet und über das Dach in die Umwelt gelassen. In den Hebeanlagen kann es durch Gärprozesse zur Gasbildung kommen. Der entstandene Unterdruck im Raum muss durch Lüftung selber geregelt werden. 
Ein anderes Phänomen ist das bei langem Transport von Fäkalien und Wasser Schwefelwasserstoff-Säure entsteht. Dies wiederum kann Metall und Turbinen angreifen. Für deren Schutz benützen Ingenieure den Duplex-Stahl mit korrosionsbeständigem Material (Keramikbeschichtung).




\paragraph{Wartung}
Abwasserrohre werden alle 30 Jahren auf Dichtigkeit kontrolliert. Die Materialien für die Abwasserrohre werden spezifisch gewählt das sie solange im Gebrauch sind wie das Haus steht. 

\subsubsection{Fazit}

Die Sicherheit für den Einbau einer Turbine in ein Abwassersystem ist gewährleistet, sofern die Normen eingehalten werden und die Verstopfungsgefahr mit einer Hebeanlage minimiert wird. 

\clearpage 





