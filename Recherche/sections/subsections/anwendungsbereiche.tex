\subsection{Anwendungsbereiche}
\subsubsection{Recherchegrund}
Wir wollen wissen, ob und welche Abwasserkraftwerksysteme bereits bestehen. Gibt es auch Systeme für die Energiegewinnung via Abwasserturbinen in Hochhäusern? Wie sieht deren Markt aus? Gibt es abgesehen vom Hochhaus noch andere Bereiche, die mit unserem System abgedeckt werden könnten?
\subsubsection{Ergebnisse}
\paragraph{Abwasserkraftwerksysteme}
Grosse Wasserkraftwerke (ca. 100\si{MW}) brauchen viel Platz (grosse Dämme und lange Kanäle) und haben einen unvorhersehbaren Effekt auf Flora und Fauna. Deshalb erscheint die Entwicklung und Realisierung von Kleinkraftwerken (ca .100\si{kW}) umso wichtiger. Die Entwicklung von Abwasserkraftwerksystemen spielt dabei eine grosse Rolle. Folgende erforschte und oder bestehende Systeme konnten gefunden werden:
\paragraph{Kanalisation zwischen Stadt und Kläranlage}
Wenn in den Abwasserrohren zwischen einer Stadt und deren Kläranlage ein hydraulisches Potential vorliegt, könnte dieses durch Turbinen genutzt werden. In Japan wurde in der Gegend des Toyogawa Flussbeckens untersucht, wo hydraulische Potentiale in den Abwasserrohren bestehen. Zudem wurden drei kleine Turbinen mit unterschiedlicher Beschaufelung auf Verstopfung durch Fremdmaterial untersucht. Die Studie (LINK) schliesst mit einem positiven Ergebnis. Ein bestehendes System konnte nicht gefunden werden.
\paragraph{Abfluss von Kläranlagen}
"Electricity is the second largest operating cost at WWTPs [Kläranlagen¨], representing 25 to 40\% of the total operating budget."
Aus diesem Grund sind viele Kläranlagen (z.Bsp. Aquarion Water Co (USA) und North Head WWTP (AUS)) darum bemüht, ihre Energiekosten zu reduzieren. In Australien wurde dies in einem Pilotprojekt bereits realisiert, indem das hydraulische Potential genutzt wurde, das zwischen der Kläranlage und dort, wo das gereinigte Wasser hinfliesst, besteht. Das dort verwendete Turbinensystem (Flow-to-Wire) (LINK) wurde von der amerikanischen Firma Rentricity hergestellt. Der Vorteil von dieser Variante ist, dass das Problem der Verstopfung duch das bereits gereinigte Wasser ausbleibt. 
\paragraph{In Hochhäusern}

\subsubsection{Fazit}
\clearpage 
