\section{Zusammenfassung}

Für den Einsatz in einem Abwasser Kleinkraftwerk wäre eine Pelton-Turbine am besten geeignet, da sie nur ein kleines Verstopfungsrisiko birgt. Da Abwasser in einer Fallleitung kaum schneller als 10 m/s wird und diese Geschwindigkeit nach einer Fallhöhe von etwa 15 Metern erreicht, könnte man in jedem fünften oder sechsten Stockwerk solch eine Turbine einbauen. Wichtig ist dabei, dass in den 15 Metern über der Turbine das Wasser nicht durch bauliche Massnahmen abgebremst wird. Die einfachste und günstigste Lösung, die gewonnene Energie ins Stromnetz einzuspeisen, wäre die Turbine in eine bestehende Solaranlage zu integrieren. Wenn das nicht möglich ist, wird ein zusätzlicher Energiespeicher oder Wechselrichter nötig, was sehr teuer ist. Der Markt für Abwasserkraftwerke ist bisher auf Grössere Anlagen mit hohem Durchfluss beschränkt, für den Einbau in Gebäuden gibt es noch keine Lösungen. Grund dafür ist ein schlechtes Preis- / Leistungsverhältnis bei kleineren Anlagen aufgrund der hohen Installationskosten und des geringen Wasserflusses.